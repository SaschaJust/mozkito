%%%%%%%%%%%%%%%%%%%%%%%%%%%%%%%%%%%%%%%%%%%%%%%%%%%%%%%%%%%%%%%%%
% Contents: Appendix - Diplomarbeit, FH Regensburg              %
%                          11.03.2003                           %
%---------------------------------------------------------------%
%                         Anhang.tex                            %
%                      by Vorname Nachname                      %
%                         mail@mail.com                         %
%%%%%%%%%%%%%%%%%%%%%%%%%%%%%%%%%%%%%%%%%%%%%%%%%%%%%%%%%%%%%%%%%

\chapter*{Abstract}
\thispagestyle {empty}
These days the analysis of issues of different bug trackers is a remarkable field of application in software mining. Given a dataset of issues of a bug tracking system like BUGZILLA, researchers mine these issues to analyze them and connect them to the source code. One extracts information from the bug reports which are used to predict failures in the concerning program for example. We choose a similar approach, but we want to pursue the mining and don't analyze or evaluate bug tracking systems or its issues itself. To be more accurately, we try to automate the mining process. In fact, we want to create a tool, whereby we are able to generate a miner or a mining plan for an arbitrary bug tracker automatically.

